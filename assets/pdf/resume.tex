\documentclass[10pt]{article}
\usepackage{multicol}
%\usepackage[cmex10]{amsmath}
\usepackage{array}
\usepackage{mdwmath}
\usepackage{wrapfig}
%\usepackage{ae}
\usepackage[T1]{fontenc}
\usepackage{tgpagella}
\usepackage{float}
\usepackage{enumitem}
%\usepackage[latin1]{inputenc}
\usepackage{mdwtab}
\usepackage{amssymb}
\usepackage{eqparbox}
\usepackage{eulervm}
\usepackage{geometry}
\usepackage{subfig}
\usepackage{graphicx}
\usepackage{setspace}
\usepackage{url}
\usepackage{hyperref}
%\usepackage{lmodern}
\geometry{lmargin=1cm,tmargin=1cm,rmargin=1cm,bmargin=1cm}
%opening
\setlength\parindent{0pt}

\begin{document}
\pagestyle{empty}
\begin{titlepage}
     {\Large{\textbf{Jos\'e Vin\'icius de Miranda Cardoso}}}
     \vspace{.5cm}

    \begin{minipage}[b]{9cm}
      keywords: data analysis, optimization, machine learning, software development, quantitative research
    \end{minipage}
    \hfill
    \begin{minipage}[b]{5cm}
        \texttt{jvdmc@connect.ust.hk}\\
        \texttt{https://mirca.github.io}\\
        \texttt{phone: +852 5444 9062}\\
        \texttt{GitHub: @mirca}
    \end{minipage}


\section*{Education}

\emph{PhD Student in Electronic and Computer Engineering} \hfill \textit{Fall 2019 -- Spring 2023 (expected)} \\
\textbf{The Hong Kong University of Science and Technology}, Hong Kong\\
\textit{Topic:} Probabilistic graphical models for financial markets\\
\textit{CGA: 3.74/4.3}\\

\emph{B.Eng. in Electrical Engineering} \hfill \textit{Class of 2019} \\
\textbf{Federal University of Campina Grande}, Brazil \\

\emph{Visiting Student -- Electrical Engineering and Computer Science} \hfill \textit{Fall 2014 -- Spring 2015} \\
\textbf{The Catholic University of America}, USA\\
\textbf{University of Maryland at College Park}, USA \\
Brazil Scientific Mobility Program, Fully funded scholarship recipient

\section*{Professional Experience}


\emph{Teaching Assistant} \hfill \textit{Feb 2020 -- Current}
\\ \textbf{The Hong Kong University of Science and Technology}, Hong Kong
  \\ {\small\textit{Courses:} Data-driven Portfolio Optimization, Convex Optimization, Portfolio Optimization with \textsf{R}}
\vspace{.5cm}

\emph{Research Scientist Intern} \hfill \textit{Summer 2021}
  \\\textbf{Shell Street Labs}, \textbf{BFAM Partners}, Hong Kong
  \\ {\small Developed \textsf{Python} tools to perform FX portfolio optimization and equity market regime identification for the}
  \\ {\small  systematic strategies team.}
\vspace{.5cm}

%\emph{Machine Learning Mentor} \hfill \textit{May 2019  -- Aug 2019}
%\\\textbf{Udacity}, Freelance, Remote
%\\ {\small Guided students from fundamental concepts of linear algebra to state-of-the-art convolutional neural nets.}
%\vspace{.5cm}

\emph{Scientific Software Engineering Intern} \hfill \textit{Mar 2017 -- Feb 2018}
\\\textbf{National Aeronautics and Space Administration}, \textbf{Ames Research Center}, Silicon Valley, CA, USA
\\Kepler/K2 Guest Observer Office
\\ {\small Developed open source \textsf{Python} code
    to assist scientists get the most out of NASA Kepler/K2/TESS time series data.}
\vspace{.5cm}

\emph{Google Summer of Code Student} \hfill \textit{Summer 2016}
\\\textbf{The AstroPy Project}
\\ Project title: Point spread function photometry for fitting overlapping stars simultaneously
\\ {\small Developed open source \textsf{Python} code
    to fit Gaussian mixture models to stellar images.}
\vspace{.5cm}

\emph{Undergraduate Guest Researcher} \hfill \textit{Summer 2015}
\\\textbf{National Institute of Standards and Technology}, USA
\\Center for Nanoscale Science and Technology
\\Nanofabrication Research Group
\\ {\small Developed \textsf{MATLAB} code to automatically localize nanoemitters in optical microscopy images.}


\section*{Volunteering Experience}
\emph{Deputy AstroPy Google Summer of Code Coordinator} \hfill \textit{Fall 2019 -- Current}
\\Deputy coordinator for the AstroPy project in the Google Summer of Code program
\\ {\small Organizing the AstroPy efforts towards participating in the Google Summer of Code.}
\vspace{.5cm}

\emph{Google Summer of Code Mentor and Organization Administrator} \hfill \textit{Summer 2018 -- Current}
\\Admin and mentor for the OpenAstronomy organization during the Google Summer of Code
\\ {\small Managing the OpenAstronomy efforts towards participating in the Google Summer of Code.}
\vspace{.5cm}


\section*{Project Proposals}
    \textbf{NASA Transiting Exoplanet Survey Satellite Proposal}
    \hfill \textit{2019}\\
    Uniform Light Curves Across the Entire Sky from TESS FFIs with ELEANOR\\
    {\small Principal Investigators: Dr. Benjamin Montet (University of Chicago) and Dr. Jacob Bean (University of Chicago)}\\

    \textbf{NASA Transiting Exoplanet Survey Satellite Proposal}
    \hfill \textit{2018}\\
    Performing The Most Comprehensive Exoplanet Survey Of The Southern Sky With TESS Full Frame Images\\
    {\small Principal Investigator: Dr. Benjamin Montet (University of Chicago)}

\section*{Selected Publications}
\begin{enumerate}
  \item \textbf{Cardoso, JVM}, Ying J, Palomar, DP. Graphical Models in Heavy-Tailed Markets.
  \textit{Advances in Neural Information Processing Systems (NeurIPS)}, Dec. 2021. Acceptance rate: 26.0\%.
  \item Ying, J, \textbf{Cardoso, JVM}, Palomar, DP. Minimax Estimation of Laplacian Constrained Precision Matrices.
    \textit{International Conference on Artificial Intelligence and Statistics (AISTATS)}, Apr. 2021. Acceptance rate: 29.8\%.
  \item Ying, J, \textbf{Cardoso, JVM}, Palomar, DP. Nonconvex Sparse Graph Learning under Laplacian-structured Graphical Model.
  \textit{Advances in Neural Information Processing Systems (NeurIPS)}, Dec. 2020. Acceptance rate: 20.1\%.
  \item \textbf{Cardoso, JVM}, Palomar, DP. Learning undirected graphs in financial markets, \textit{54th Asilomar Conference on Signals, Systems, and Computers}, Sept. 2020.
  \item Kumar, S, Ying, J, \textbf{Cardoso, JVM}, Palomar, DP. A unified framework for structured graph
    learning via spectral constraints. \textit{Journal of Machine Learning Research (JMLR)}, Jan. 2020.
  \item Kumar, S, Ying, J, \textbf{Cardoso, JVM}, Palomar, DP. Structured graph learning via Laplacian
    spectral constraints. \textit{Advances in Neural Information Processing Systems (NeurIPS)}, Dec. 2019. Acceptance rate: 21.6\%.
  \item \textbf{Lightkurve Collaboration}, \textit{et. al.} Lightkurve: Kepler and TESS time series analysis in Python. \textit{Astrophysics Source Code Library}, 2018.
  \item Price-Whelan, AM, \textit{et. al.} The Astropy Project: Building an open-science project and status of the v2.0 Core Package, \textit{The Astronomical Journal 156}, 2018.
\item Davanco, MI, Liu, J, Sapienza, L, Zhang, CZ, \textbf{Cardoso, JVM}, Verma, V, Mirin, R, Nam,
SW, Srinivasan, K. Heterogeneous integration for on-chip quantum photonic circuits with single quantum dot devices.
\textit{Nature Communications}, 2017.
\end{enumerate}

For a complete list of my publications, please refer to my Google Scholar profile \url{https://scholar.google.com/citations?user=ilvNpCoAAAAJ&hl=en}.

\section*{Awards}
\begin{enumerate}
  \item Full travel funding to attend the workshop \textit{Preparing for TESS}, Flatiron Institute, New York City, USA, 2018
  \item Selected to the workshop \textsf{Python} in Astronomy, Leiden, The Netherlands, 2017
  \item Selected, with full travel funding, to the S\~ao Paulo School of Advanced Science on Nanophotonics, S\~ao Paulo, Brazil, 2016
  \item Travel Grant (U\$ 2000,00) by the Institute of Electronic and Electrical Engineers to attend the IEEE APS Meeting 2016
\end{enumerate}


\section*{Competencies}
\begin{description}
  \item[Coding:] \textsf{Python}, \textsf{R}, \textsf{C/C++}, git/GitHub, Unix shell, unit tests, continuous integration/development
  \item[Courses:] Convex Optimization, Stochastic Processes, Deep Learning Architectures, Topological and Geometric Data Analysis
\end{description}

\section*{Open-source Projects on GitHub}

  \begin{itemize}
    \item[$\triangleright$] \textsf{spectralGraphTopology}: learning graphs from data (40+ stars on GitHub)
    \item[$\triangleright$] \textsf{riskParityPortfolio}: design of risk parity portfolios in $\textsf{R}$ (60+ stars on GitHub)
    \item[$\triangleright$] \textsf{riskparity.py}: optimization of risk parity portfolios in $\textsf{Python}$ (160+ stars on GitHub)
  \end{itemize}

\end{titlepage}

\end{document}
